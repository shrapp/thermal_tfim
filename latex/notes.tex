\documentclass{article}
\usepackage{amsmath}
\usepackage{amssymb}
\usepackage{graphicx}
\usepackage{geometry}
\usepackage{breqn}
\geometry{margin=1in}
\title{TFIM Noisy States as Thermal States}
\begin{document}
\maketitle

\section*{Sections:}
\begin{enumerate}
    \item Motivation for the paper: create a benchmark for NISQ era computers using temperature as a single parameter to describe the amount of noise in the system.
    \item The paper is about the TFIM model with specific linear quench protocol (add the whole derivation)
    \item What is known about the TFIM model without noise (add some graphs of system size vs tau)
    \item The main sub sections of the analytic and numeric process and their contribution to the final states
    \item How thermal states are defined and behave in this model
    \item How the noise is defined and implemented in the model
    \item How noise affects the system in the Ks space and in the real space
    \item Thermal vs noise: not the same
    \item Summary
\end{enumerate}

\section*{Motivation}
This paper came from the will to create simple NISQ era computer benchmark using temperature as a single parameter to describe the amount of noise in the system. We have tried to describe the states of the system at the end of the process as thermal states.

\section*{The Model}
In this study, we are looking into a transverse field Ising model, which is going through a quantum phase transition. The Hamiltonian is:
\begin{equation}
    \hat{H}(t)=-J\sum_{n=1}^N \left( g(t) \cdot \sigma_n^z \sigma_{n+1}^z + (1-g(t)) \sigma_n^x \right)
\end{equation}
with periodic boundary conditions: $\sigma_{N+1}=\sigma_1$ and $g(t)=\frac{t}{\tau}+\gamma(t)$ while $0 \leq t \leq \tau$. For convenience we assume $N$ is even.
\\After a Jordan-Wigner transformation defining: $\sigma_n^x=1-2c_n^\dagger c_n$ and $\sigma_n^z=-(c_n^\dagger+c_n) \prod_{m<n} (1-2c_m^\dagger c_m)$
\begin{equation}
    \begin{split}
        \hat{H}(t) & =-J\sum_{n=1}^N \bigg( g(t) \cdot \bigg( -(c_n^\dagger+c_n) \prod_{m<n} (1-2c_m^\dagger c_m) \bigg) \bigg( -(c_{n+1}^\dagger+c_{n+1}) \prod_{m<n+1} (1-2c_m^\dagger c_m) \bigg) \\
        & \qquad + (1-g(t))(1-2c_n^\dagger c_n) \bigg) \\
        & =-J\sum_{n=1}^N \bigg( g(t) \cdot (c_n^\dagger+c_n) \bigg( \prod_{m<n} (1-2c_m^\dagger c_m) \bigg)^2 (c_{n+1}^\dagger+c_{n+1})(1-2c_n^\dagger c_n) \\
        & \qquad + (1-g(t))(1-2c_n^\dagger c_n) \bigg) \\
        & =-J\sum_{n=1}^N \bigg( g(t)(c_n^\dagger+c_n)(c_{n+1}^\dagger+c_{n+1}) + (1-g(t))(1-2c_n^\dagger c_n) \bigg) \\
        & =-J\sum_{n=1}^N \bigg( g(t) c_n^\dagger c_{n+1}^\dagger +g(t) c_n^\dagger c_{n+1}+g(t) c_n c_{n+1}^\dagger +g(t) c_n c_{n+1}+1-g(t) \\
        & \qquad -2g(t) c_n^\dagger c_{n+1}^\dagger c_n^\dagger c_n-2g(t) c_n^\dagger c_{n+1} c_n^\dagger c_n-2g(t) c_n c_{n+1}^\dagger c_n^\dagger c_n-2c_n^\dagger c_n+2g(t) c_n^\dagger c_n \bigg) \\
        & =-J\sum_{n=1}^N \bigg( g(t)(c_n^\dagger c_{n+1}^\dagger+c_n c_{n+1}) +g(t)(c_n^\dagger c_{n+1}+c_n c_{n+1}^\dagger )+1-g(t) \\
        & \qquad -2g(t) c_n c_{n+1}^\dagger c_n^\dagger c_n-2g(t) c_n c_{n+1} c_n^\dagger c_n-2c_n^\dagger c_n+2g(t) c_n^\dagger c_n \bigg) \\
        & =-J\sum_{n=1}^N \bigg( g(t)(c_n^\dagger c_{n+1}^\dagger-c_n c_{n+1}) +g(t)(c_n^\dagger c_{n+1}-c_n c_{n+1}^\dagger )+1-g(t) \\
        & \qquad -(1-g(t))(c_n^\dagger c_n+1-c_n c_n^\dagger ) \bigg) \\
        & =-J\sum_{n=1}^N \bigg( g(t)(c_n^\dagger c_{n+1}^\dagger-c_n c_{n+1}) +g(t)(c_n^\dagger c_{n+1}-c_n c_{n+1}^\dagger )-(1-g(t))(c_n^\dagger c_n-c_n c_n^\dagger ) \bigg)
    \end{split}
\end{equation}
we can write the Hamiltonian as: $\hat{H}(t)=P^\dagger \hat{H}^\dagger (t) P^\dagger+P^- \hat{H}^- (t) P^-$ where : $P^\pm=\frac{1}{2} (1\pm\prod_{n=1}^N (1-2c_n^\dagger c_n) )$. we can confine ourselves to the subspace of even parity because we start at the ground state which has even parity for any value of $g$. In that subspace: $P^-=0,P^\dagger=1$ s.t. $\hat{H}(t)=P^\dagger \hat{H}^\dagger (t) P^\dagger=\hat{H}^\dagger (t)=-J\sum_{n=1}^N \left( g(t)(c_n^\dagger c_{n+1}^\dagger-c_n c_{n+1}) +g(t)(c_n^\dagger c_{n+1}-c_n c_{n+1}^\dagger )-(1-g(t))(c_n^\dagger c_n-c_n c_n^\dagger ) \right)$ and $c_n$’s in $\hat{H}^\dagger$ must obey $c_{N+1}=-c_1$. $\hat{H}^\dagger$ is diagonalized by Fourier transform followed by Bogoliubov transformation: $c_n=\frac{e^{-i \pi/4}}{\sqrt{N}} \sum_k c_k e^{ikn}$ , and $k=\pm\frac{z\pi}{N} \, \text{for} \, z \in 1,2,\ldots,N-1$.
\[
    c_n^\dagger c_{n+1}^\dagger=\frac{e^{i \pi/4}}{\sqrt{N}} \sum_{k'} c_{k'}^\dagger e^{-ik'n} \frac{e^{i \pi/4}}{\sqrt{N}} \sum_k c_k^\dagger e^{-ik(n+1)} = \frac{e^{i 2\pi/4}}{N} \sum_{k,k'} e^{-ik} c_k^\dagger c_{k'}^\dagger e^{-in(k+k')}
\]
\[
    c_n c_{n+1}=\frac{e^{-i \pi/4}}{\sqrt{N}} \sum_k c_k e^{ikn} \frac{e^{-i \pi/4}}{\sqrt{N}} \sum_{k'} c_{k'} e^{ik'(n+1)} = \frac{e^{-i 2\pi/4}}{N} \sum_{k,k'} e^{-ik'} c_k c_{k'} e^{-in(k+k')}
\]
\[
    c_n^\dagger c_{n+1}=\frac{e^{i \pi/4}}{\sqrt{N}} \sum_{k'} c_{k'}^\dagger e^{-ik'n} \frac{e^{-i \pi/4}}{\sqrt{N}} \sum_k c_k e^{ik(n+1)} = \frac{1}{N} \sum_{k,k'} e^{-ik} c_{k'}^\dagger c_k e^{-in(k-k')}
\]
\begin{equation}
    \begin{aligned}
        \hat{H}^+(t) & = -J\sum_{n=1}^N \left(g(t)(c_n^\dagger c_{n+1}^\dagger - c_n c_{n+1}) + g(t)(c_n^\dagger c_{n+1} - c_n c_{n+1}^\dagger) - (1-g(t))(c_n^\dagger c_n - c_n c_n^\dagger)\right) \\
                     & = -J\sum_k g(t)(-\sin(k) (c_k^\dagger c_{-k}^\dagger - c_k c_{-k}) + \cos(k) (c_k^\dagger c_k - c_{-k} c_{-k}^\dagger))                                                       \\
                     & \quad - (1-g(t))(c_k^\dagger c_k - c_{-k} c_{-k}^\dagger)                                                                                                                     \\
                     & = J\sum_k \left(g(t) \sin(k) (c_k^\dagger c_{-k}^\dagger + c_{-k} c_k) + (1-g(t)-g(t)\cos(k))(c_k^\dagger c_k - c_{-k} c_{-k}^\dagger)\right)                                 \\
                     & = J\sum_k \begin{pmatrix}
                                     c_k^\dagger & c_{-k}
                                 \end{pmatrix}
        \begin{pmatrix}
            (1-g-g \cos(k)) & g \sin(k)        \\
            g \sin(k)       & -(1-g-g \cos(k))
        \end{pmatrix}
        \begin{pmatrix}
            c_k \\
            c_{-k}^\dagger
        \end{pmatrix}                                                                                                                                                                               \\
                     & = 2J\sum_{k>0} \left(g(t) \sin(k) (c_k^\dagger c_{-k}^\dagger + c_{-k} c_k) + (1-g(t)-g(t)\cos(k))(c_k^\dagger c_k - c_{-k} c_{-k}^\dagger)\right)                            \\
                     & \quad + 2J\sum_{k>0} \left(-g(t) \sin(k) (c_{-k}^\dagger c_k^\dagger + c_k c_{-k}) + (1-g(t)-g(t)\cos(k))((1-c_{-k} c_{-k}^\dagger) - (1-c_k^\dagger c_k))\right)             \\
                     & = 2J\sum_{k>0} \left(g(t) \sin(k) (c_k^\dagger c_{-k}^\dagger + c_{-k} c_k) + (1-g(t)-g(t)\cos(k))(c_k^\dagger c_k - c_{-k} c_{-k}^\dagger)\right)                            \\
                     & \quad + 2J\sum_{k>0} \left(g(t) \sin(k) (c_k^\dagger c_{-k}^\dagger + c_{-k} c_k) + (1-g(t)-g(t)\cos(k))((1-c_{-k} c_{-k}^\dagger) - (1-c_k^\dagger c_k))\right)              \\
                     & = 2J\sum_{k>0}
        \begin{pmatrix}
            c_k^\dagger & c_{-k}
        \end{pmatrix}
        \begin{pmatrix}
            (1-g-g \cos(k)) & g \sin(k)        \\
            g \sin(k)       & -(1-g-g \cos(k))
        \end{pmatrix}
        \begin{pmatrix}
            c_k \\
            c_{-k}^\dagger
        \end{pmatrix}                                                                                                                                                                               \\
                     & = 2J\sum_{k>0} \left(g(t) \sin(k) (c_k^\dagger c_{-k}^\dagger + c_{-k} c_k) + (1-g(t)-g(t)\cos(k))(c_k^\dagger c_k - c_{-k} c_{-k}^\dagger)\right).
    \end{aligned}
\end{equation}

Now, let's define: \(c_k = u_k(t) \gamma_k + v_{-k}^*(t) \gamma_{-k}^\dagger\) or equivalently \(\gamma_k = u_k^* c_k + v_{-k} c_{-k}^\dagger\) where \(\gamma_k\) annihilates a fermionic Bogoliubov quasiparticle. And \((u_k, v_k)\) are eigenstates of the stationary Bogoliubov-de Gennes equations:

\[
    \omega_k
    \begin{pmatrix}
        u_k \\
        v_k
    \end{pmatrix}
    =
    \begin{pmatrix}
        (1-g-g \cos(k)) & g \sin(k)        \\
        g \sin(k)       & -(1-g-g \cos(k))
    \end{pmatrix}
    \begin{pmatrix}
        u_k \\
        v_k
    \end{pmatrix},
\]

\[
    \omega_k = \pm\sqrt{(1-g-g \cos(k))^2 + g^2 \sin^2 k}.
\]

Substitute into \(\hat{H}^+\) will yield \(\hat{H}^+ = 2J\sum_k \omega_k (\gamma_k^\dagger \gamma_k - 1/2)\).

The number of kinks operator is \(N = \frac{1}{2} \sum_{n=1}^N (1 - \sigma_n^Z \sigma_{n+1}^Z)\) , when \(g(t)=1\) at the end of the process:

\[
    2J\sum_k (\gamma_k^\dagger \gamma_k - 1/2) = -J\sum_{n=1}^N \sigma_n^Z \sigma_{n+1}^Z = J(2N - N) = J(2\sum_k \gamma_k^\dagger \gamma_k - N)
\]

\[
    2N - N = 2\sum_k \gamma_k^\dagger \gamma_k - N \Rightarrow N = \sum_k \gamma_k^\dagger \gamma_k.
\]
\begin{align*}
    \langle \gamma_k^\dagger \gamma_k \rangle & = \langle (u_k^{* \dagger} c_k^\dagger + v_{-k}^\dagger c_{-k})(u_k^* c_k + v_{-k} c_{-k}^\dagger) \rangle                                                                                                                                   \\
                                              & = \langle (u_k^{* \dagger} u_k^* c_k^\dagger c_k + u_k^{* \dagger} v_{-k} c_k^\dagger c_{-k}^\dagger + v_{-k}^\dagger u_k^* c_{-k} c_k + v_{-k}^\dagger v_{-k} c_{-k} c_{-k}^\dagger) \rangle                                                \\
                                              & = u_k^{* \dagger} u_k^* \langle c_k^\dagger c_k \rangle + u_k^{* \dagger} v_{-k} \langle c_k^\dagger c_{-k}^\dagger \rangle + v_{-k}^\dagger u_k^* \langle c_{-k} c_k \rangle + v_{-k}^\dagger v_{-k} \langle c_{-k} c_{-k}^\dagger \rangle.
\end{align*}

For \( \cos(k) \geq 0 \) the matrix
\[
    \begin{pmatrix}
        g - \cos(k) & \sin(k)        \\
        \sin(k)     & -(g - \cos(k))
    \end{pmatrix}
\]
can be described by the Landau-Zener formula such that \( \langle \gamma_k^\dagger \gamma_k \rangle \propto e^{-J/\hbar 2\pi\tau \sin^2(k)} \) but otherwise the modes don't complete the Landau-Zener transition.

For \( \cos(k) < 0 \), \( (g - \cos(k)) \) never reaches 0 and there isn't any level crossing such that the end state doesn’t depend on \( \tau \) and can be determined by the projection of the final states.

\begin{align*}
    H_{\text{lz}}(t=0)      & = 2\sum_{k>0}
    \begin{pmatrix}
        c_k^\dagger & c_{-k}
    \end{pmatrix}
    \begin{pmatrix}
        -\cos(k) & \sin(k) \\
        \sin(k)  & \cos(k)
    \end{pmatrix}
    \begin{pmatrix}
        c_k \\
        c_{-k}^\dagger
    \end{pmatrix},                                                               \\
    \Rightarrow \lambda_\pm & = \pm 1,                                           \\
    \Rightarrow \nu_\pm     & = \frac{1}{\sqrt{((\cos(k) \pm 1)/\sin(k))^2 + 1}}
    \begin{pmatrix}
        1 \\
        (\cos(k) \pm 1)/\sin(k)
    \end{pmatrix}
    =
    \begin{pmatrix}
        \sin(k/2) \\
        \cos(k/2)
    \end{pmatrix}.
\end{align*}


\[
    \left| \langle \psi | \nu_+ \rangle \right|^2 = \left| c_- \sin(k/2) + c_+ \cos(k/2) \right|^2 = f(k, \tau) \sim e^{-g(k,\tau)} \cos^2(k/2),
\]

where

\[
    g(k, \tau \ll 1) = 16.6\tau \sin^2(k/2), \quad g(k, \tau = 0) = 0.
\]

\[
    \left| \langle - | \nu_+ \rangle \right|^2 = \left|
    \begin{pmatrix}
        0 & 1
    \end{pmatrix}
    \frac{1}{\sqrt{((\cos(k) + 1)/\sin(k))^2 + 1}}
    \begin{pmatrix}
        1 \\
        (\cos(k) + 1)/\sin(k)
    \end{pmatrix}
    \right|^2
\]
\[
    = \left| \frac{(\cos(k) + 1)/\sin(k)}{\sqrt{((\cos(k) + 1)/\sin(k))^2 + 1}} \right|^2 = \cos^2(k/2).
\]

By diagonalizing this matrix at \( g = 0 \) we get that the overlap between the initial minus eigenstate and the final excited eigenstate is \( \cos^2(k/2) \). It works in both boundaries when \( k = 0 \), and for \( k = \pi \). So we can assume that the total excitation probability can be described as the product:

\[
    \langle \gamma_k^\dagger \gamma_k \rangle \sim e^{-J/\hbar 2\pi\tau \sin^2(k)} \cdot \cos^2(k/2) = p_k.
\]

\[
    \delta[\hat{N} - n] = \frac{1}{2\pi} \int_{-\pi}^{\pi} d\theta e^{i\theta(\hat{N} - n)}.
\]

\[
    P(n) = \langle \delta[\hat{N}-n] \rangle = \mathrm{Tr}[\rho_{\tau} \delta[\hat{N}-n]] = \mathrm{Tr}[\rho_{\tau} \frac{1}{2\pi} \int_{-\pi}^{\pi} d\theta e^{i\theta(\hat{N}-n)}] = \frac{1}{2\pi} \int_{-\pi}^{\pi} d\theta \mathrm{Tr}[\rho_{\tau} e^{i\theta\hat{N}}] e^{-i\theta n}.
\]

\[
    \mathrm{Tr}[\rho_{\tau} e^{i\theta\hat{N}}] = \tilde{P}(\theta,\tau) = \chi(\theta)
\]
\[
    \hat{N} = \sum_k (\gamma_k^{\dagger} \gamma_k)
\]
\[
    \rho_{\tau} = \bigotimes \rho_{(k,\tau)}
\]
\[
    \tilde{P}(\theta,\tau) = \prod_k \mathrm{Tr}[\rho_{(k,\tau)} e^{i\theta\gamma_k^{\dagger} \gamma_k}].
\]

\[
    e^{i\theta\gamma_k^{\dagger} \gamma_k} = \sum_{i=0}^{\infty} \frac{(i\theta)^i}{i!} (\gamma_k^{\dagger} \gamma_k)^i = I + \sum_{i=1}^{\infty} \frac{(i\theta)^i}{i!} (\gamma_k^{\dagger} \gamma_k)^i = I + \gamma_k^{\dagger} \gamma_k \sum_{i=1}^{\infty} \frac{(i\theta)^i}{i!} = I + \gamma_k^{\dagger} \gamma_k (e^{i\theta}-1).
\]

\[
    P(n) = \langle \delta[\hat{N}-n] \rangle = \frac{1}{2\pi} \int_{-\pi}^{\pi} d\theta \prod_k [1+(e^{i\theta}-1)\langle \gamma_k^{\dagger} \gamma_k \rangle] e^{-i\theta n}.
\]

\[
    \tilde{P}(\theta,\tau) = \prod_k \mathrm{Tr}[\rho_{(k,\tau)} e^{i\theta\gamma_k^{\dagger} \gamma_k}] = \prod_k [1+(e^{i\theta}-1)\langle \gamma_k^{\dagger} \gamma_k \rangle]
\]
\[
    \langle \gamma_k^{\dagger} \gamma_k \rangle \sim e^{-(2\pi J\tau\sin^2(k))/\hbar} \cos^2(k/2) = p_k.
\]

\begin{equation}
    \ln(\tilde{P}(\theta,\tau)) = \sum_k \ln(1+(e^{i\theta}-1)p_k)
\end{equation}
\begin{equation}
    \sum_{k>0}\ln(1+(e^{i2\theta}-1)p_k)
\end{equation}
\begin{equation}
    K(\theta) = \ln(\chi(\theta)) = \ln\left(\prod_k [1+(e^{i\theta}-1)\langle \gamma_k^{\dagger} \gamma_k \rangle]\right) = \sum_k \ln(1+(e^{i\theta}-1)p_k).
\end{equation}

\begin{align*}
    \text{In the continuum limit:} \, & \sum_{k>0}\ln(1+(e^{2i\theta}-1) p_k ) \approx \frac{N}{\pi} \int_0^{\pi} dk \, \ln(1+(e^{2i\theta}-1) p_k )                                                                              \\
                                      & \approx \frac{N}{\pi} \int_0^{\pi} dk \, \sum_{p=1}^{\infty} \frac{(-1)^{p+1}}{p} ((e^{2i\theta}-1) p_k )^p                                                                               \\
                                      & = -\frac{N}{\pi} \sum_{p=1}^{\infty}\frac{(1-e^{2i\theta})^p}{p} \int_0^{\pi} dk \, \left(e^{-\left(\frac{2\pi J\tau \sin^2 (k)}{\hbar}\right)} \cos^2 \left(\frac{k}{2}\right)\right)^p.
\end{align*}

\text{For } $\tau > 10^{-1} : \, e^{-\left(\frac{2\pi J\tau \sin^2 (k)}{\hbar}\right)} \cos^2 \left(\frac{k}{2}\right) \approx e^{-\left(\frac{2\pi J\tau p k^2}{\hbar}\right)}$

\begin{align*}
    -\frac{N}{\pi} \sum_{p=1}^{\infty}\frac{(1-e^{2i\theta})^p}{p} \int_0^{\pi} dk \, e^{-\left(\frac{2\pi J\tau p k^2}{\hbar}\right)} & = -\frac{N}{2\pi} \sum_{p=1}^{\infty}\frac{(1-e^{2i\theta})^p}{p} \sqrt{\frac{\hbar}{2J\tau p}} \, \text{erf}\left(\sqrt{\frac{2J p \tau \pi^3}{\hbar}}\right) \\
                                                                                                                                       & = \sum_{p=1}^{\infty} \frac{(i\theta)^p}{p!} \kappa_p.
\end{align*}

\text{For } $\tau > \frac{1}{3}$, $\text{erf}\left(\sqrt{\frac{2J p \tau \pi^3}{\hbar}}\right) \approx 1$ \, \text{ such that }

\begin{align*}
    -\frac{N}{2\pi} \sum_{n=1}^{\infty} \left(-\sum_{m=1}^{\infty}\frac{(2i\theta)^m}{m!}\right)^n/n \sqrt{\frac{\hbar}{2J\tau n}}   & = \sum_{p=1}^{\infty} \frac{(i\theta)^p}{p!} \kappa_p  \\
    -\frac{N}{2\pi} \sum_{n=1}^{\infty} \frac{((-1)^n (2i\theta-2\theta^2-(8i\theta^3)/6\ldots)^n)}{n} \sqrt{\frac{\hbar}{2J\tau n}} & = \sum_{p=1}^{\infty} \frac{(i\theta)^p}{p!} \kappa_p.
\end{align*}

\begin{align*}
    \kappa_2 & = \frac{2}{(-\theta^2)} \left(-\frac{N}{2\pi}\right)(2\theta^2 \sqrt{\frac{\hbar}{2J\tau}}+(2i\theta)^2/2 \sqrt{\frac{\hbar}{4J\tau}}) \sqrt{\frac{\hbar}{2J\tau n}}=\kappa_1 (2-\sqrt{2})  \\
    \text{Which yields:}                                                                                                                                                                                   \\
    \kappa_1 & = \sqrt{\frac{\hbar}{2J\tau}} \frac{N}{\pi},                                                                                                                                                \\
    \kappa_2 & = \frac{2}{(-\theta^2)} \left(-\frac{N}{2\pi}\right)(2\theta^2 \sqrt{\frac{\hbar}{2J\tau}}+(2i\theta)^2/2 \sqrt{\frac{\hbar}{4J\tau}}) \sqrt{\frac{\hbar}{2J\tau n}}=\kappa_1 (2-\sqrt{2}), \\
    \kappa_3 & = \kappa_1 2(1-3/\sqrt{2}+2/\sqrt{3}).
\end{align*}

\text{For small } $\tau$:

\begin{align*}
    -\frac{N}{\pi} \sum_{p=1}^{\infty}\frac{(1-e^{2i\theta})^p}{p} \int_0^{\pi} dk \, \left(e^{-\left(\frac{2\pi J\tau \sin^2 (k)}{\hbar}\right)} \cos^2 \left(\frac{k}{2}\right)\right)^p \\
     & \approx -\frac{N}{\sqrt{\pi}} \sum_{p=1}^{\infty} \left(-\sum_{m=1}^{\infty}\frac{(2i\theta)^m}{m!}\right)^p \frac{1}{p} \frac{\Gamma\left(p+\frac{1}{2}\right)}{\Gamma(p+1)}       \\
     & = \sum_{p=1}^{\infty} \frac{(i\theta)^p}{p!} \kappa_p.
\end{align*}

\begin{align*}
    \kappa_p & = -\frac{N p!}{(\sqrt{\pi} (i\theta)^p)} \sum_{n=1}^{\infty} \frac{((-1)^n (\sum_{m=1}^{\infty}(2i\theta)^m/m!)^n)}{n} \, \frac{\Gamma\left(n+\frac{1}{2}\right)}{\Gamma(n+1)} : i\theta \text{ cancel}      \\
    \kappa_p & = -\frac{N p!}{(\sqrt{\pi} (i\theta)^p)} \sum_{n=1}^{\infty} \frac{((-1)^n (2i\theta-2\theta^2-(8i\theta^3)/6\ldots)^n)}{n} \, \frac{\Gamma\left(n+\frac{1}{2}\right)}{\Gamma(n+1)} : i\theta \text{ cancel} \\
    \kappa_1 & = -\frac{N}{(\sqrt{\pi} (i\theta)^1)} \frac{((-1)^1 (2i\theta-2\theta^2-(8i\theta^3)/6\ldots)^1)}{1} \, \sqrt{\pi}/2=N                                                                                       \\
    \kappa_2 & = -\frac{N 2}{(\sqrt{\pi} (i\theta)^2)} (\theta^2 \sqrt{\pi}+(-4\theta^2)/2 \, 3 \sqrt{\pi}/8)=\kappa_1 2(1-3/4)=\kappa_1/2.
\end{align*}
For large $\tau$:
\begin{align*}
    \sum_{k>0} \ln(1+(e^{2i\theta}-1) p_k) & = \sum_{k>0} \ln\left(1+e^{-\left(\frac{2\pi J\tau \sin^2 (k)}{\hbar}\right)} \cos^2 \left(\frac{k}{2}\right) \sum_{m=1}^{\infty} \frac{(2i\theta)^m}{m!}\right)   \\
                                           & \approx \ln\left(1+e^{-\left(\frac{2\pi J\tau \sin^2 (\pi/N)}{\hbar}\right)} \cos^2 \left(\frac{\pi}{2N}\right) \sum_{m=1}^{\infty} \frac{(2i\theta)^m}{m!}\right)
\end{align*}
\[
    = \sum_{p=1}^{\infty} \left(\frac{-1^{p+1}}{p}\right) \left(\left(\sum_{m=1}^{\infty} \frac{(2i\theta)^m}{m!}\right) e^{-\left(\frac{2\pi J\tau \sin^2 (\pi/N)}{\hbar}\right)} \cos^2 \left(\frac{\pi}{2N}\right)\right)^p = \sum_{p=1}^{\infty} \frac{(i\theta)^p}{p!} \kappa_p
\]
\[
    \kappa_1 = 2e^{-\left(\frac{2\pi J\tau \sin^2 (\pi/N)}{\hbar}\right)} \cos^2 \left(\frac{\pi}{2N}\right)
\]
\[
    \kappa_2 = 4e^{-\left(\frac{2\pi J\tau \sin^2 (\pi/N)}{\hbar}\right)} \cos^2 \left(\frac{\pi}{2N}\right) - 4e^{-\left(\frac{4\pi J\tau \sin^2 (\pi/N)}{\hbar}\right)} \cos^4 \left(\frac{\pi}{2N}\right) \approx 2\kappa_1
\]

\[
    p_k(k, \tau) \approx e^{-f(k,\tau)} \cos^2 \left(\frac{k}{2}\right)
\]

\[
    f(k, \tau \ll 1) \approx 16.6\tau \sin^2 \left(\frac{k}{2}\right)
\]

\section*{Thermal description of the Ising model}
The energy of the Ising model is (up to a constant)
\[
    H = 2J \sum d_i
\]
Where $d_i$ is a binary variable (0/1) signaling whether there's a kink at bond $i$. In a thermal state:
\[
    P(d_i) = \frac{Z \exp(-2J\beta d_i)}{1 + \exp(-2J \beta)} = \frac{\exp(-2J \beta d_i)}{1 + \exp(-2J \beta)}
\]

The average density of defects is then for $J=1$:
\[
    \langle d_i \rangle = \frac{e^{-2\beta}}{1 + e^{-2\beta}} \Rightarrow e^{-2\beta}(1-d) = d \Rightarrow \beta = -\frac{1}{2} \log\left(\frac{d}{1-d}\right)
\]

Thermal 1:
\[
    P(\text{kinks}=n) \propto \binom{N}{n} \left(\frac{e^{-2\beta}}{1 + e^{-2\beta}}\right)^n \left(\frac{1}{1 + e^{-2\beta}}\right)^{N-n} = \frac{1}{\text{Norm}} \frac{e^{-2n\beta}}{n!(N-n)!}
\]
\[
    \sum_{n \text{ even}} \frac{e^{-2\beta n}}{n!(N-n)!} = \text{Norm}
\]

Thermal 2:
\[
    P(\text{kinks}=n) \propto \binom{N/2}{n/2} \left(\frac{e^{-2\beta}}{1 + e^{-2\beta}}\right)^n \left(\frac{1}{1 + e^{-2\beta}}\right)^{N-n} = \frac{1}{\text{Norm}} \frac{e^{-2n\beta}}{(n/2)!(N/2-n/2)!}
\]
\[
    \sum_{n \text{ even}} \frac{e^{-2\beta n}}{n!(N-n)!} = \text{Norm}
\]

For thermal states:
\[
    M(t) = \frac{\sum_{n \text{ even}} e^{n(t-2\beta)} / n!(N-n)!}{\text{Norm}}
\]
\[
    K(t) = \log(M(t)) = \log\left(\frac{\sum_{n \text{ even}} e^{n(t-2\beta)} / n!(N-n)!}{\text{Norm}}\right) = \log\left(\sum_{n \text{ even}} \frac{e^{n(t-2\beta)}}{n!(N-n)!}\right) - \log(\text{Norm})
\]
\[
    \kappa_1 = \left. \frac{dK}{dt} \right|_{t=0} = \frac{1}{\text{Norm}} \frac{d}{dt} \left(\sum_{n \text{ even}} \frac{e^{n(t-2\beta)}}{n!(N-n)!}\right) = \frac{\sum_{n \text{ even}} \frac{ne^{-2\beta n}}{n!(N-n)!}}{\text{Norm}}
\]
\begin{align*}
    \kappa_2 & = \left. \frac{d^2 K}{dt^2} \right|_{t=0}                                                                                                                                                                                                                                                                               \\
             & = \frac{d}{dt} \left(\frac{\sum_{n \text{ even}} \frac{ne^{n(t-2\beta)}}{n!(N-n)!}}{\sum_{n \text{ even}} \frac{e^{n(t-2\beta)}}{n!(N-n)!}}\right)                                                                                                                                                                      \\
             & = \left.\frac{\left(\sum_{n \text{ even}} \frac{n^2 e^{n(t-2\beta)}}{n!(N-n)!}\right) \left(\sum_{n \text{ even}} \frac{e^{n(t-2\beta)}}{n!(N-n)!}\right) - \left(\sum_{n \text{ even}} \frac{ne^{n(t-2\beta)}}{n!(N-n)!}\right)^2}{\left(\sum_{n \text{ even}} \frac{e^{n(t-2\beta)}}{n!(N-n)!}\right)^2}\right|_{t=0} \\
             & = \frac{\left(\sum_{n \text{ even}} \frac{n^2 e^{-2\beta n}}{n!(N-n)!}\right)}{\text{Norm}} - \kappa_1^2
\end{align*}
\[
    \frac{\kappa_2}{\kappa_1} = \frac{\left(\sum_{n \text{ even}} \frac{n^2 e^{-2\beta n}}{n!(N-n)!}\right)}{\left(\sum_{n \text{ even}} \frac{ne^{-2\beta n}}{n!(N-n)!}\right)} - \frac{\left(\sum_{n \text{ even}} \frac{ne^{-2\beta n}}{n!(N-n)!}\right)}{\left(\sum_{n \text{ even}} \frac{e^{-2\beta n}}{n!(N-n)!}\right)}
\]

\begin{align*}
    \text{For small } \tau & : d \rightarrow 0.5 \Rightarrow \beta \rightarrow 0                                                                                                                                                                                                                                          \\
    \Rightarrow \kappa_1   & = \left\{ \begin{aligned}
                                            & \frac{-\left((1 - e^{-2 b})^N  + (e^{-2 b}  + 1)^N  + e^{2 b}  ((1 - e^{-2 b})^N  - (e^{-2 b}  + 1)^N )\right)}{2 (e^{4 b}  - 1)(N - 1)!} \\
                                            & \times \left(\frac{(1-e^{-2b})^N+(1+e^{-2b})^N}{2N!}\right)^{-1}
                                       \end{aligned} \right. \\
                           & \rightarrow \left\{ \begin{aligned}
                                                      & \frac{-\left((e^{-2 b}  + 1)^N  + e^{2 b}  (- (e^{-2 b}  + 1)^N )\right)}{2 (e^{4 b}  - 1)(N - 1)!} \\
                                                      & \times \left(\frac{(1+e^{-2b})^N}{2N!}\right)^{-1}
                                                 \end{aligned} \right.                                                                         \\
                           & = \left\{ \begin{aligned}
                                            & \frac{-( 2^N- e^{2 b}  2^N)}{2 (e^{4 b}  - 1)(N - 1)!} \\
                                            & \times \left(\frac{2^N}{2N!}\right)^{-1}
                                       \end{aligned} \right.                                                                                                                                                                            \\
                           & = \left\{ \begin{aligned}
                                            & \frac{(2^N (e^{2 b}-1))}{2 (e^{4 b}  - 1)(N - 1)!} \\
                                            & \times \left(\frac{2^N}{2N!}\right)^{-1}
                                       \end{aligned} \right.                                                                                                                                                                                \\
                           & = \frac{2^N}{2 (e^{2b}+1)(N - 1)!} \left(\frac{2^N}{2N!}\right)^{-1} = \frac{N!}{2(N - 1)!}  = \frac{N}{2}.
\end{align*}



\begin{align*}
    \kappa_2 & = \frac{1}{2(e^{-4\beta}-1)^2 (N-1)!} \left(e^{6\beta} ((1+e^{-2\beta} )^N-(1-e^{-2\beta} )^N ) + e^{4\beta} ((1+e^{-2\beta} )^N+(1-e^{-2\beta} )^N )(N-2) \right. \\
             & \left. + N((1+e^{-2\beta} )^N+(1-e^{-2\beta} )^N ) + e^{2\beta} ((1-e^{-2\beta} )^N-(1+e^{-2\beta} )^N )(2N-1) \right)                                             \\
             & \left(\frac{(1-e^{-2b})^N+(1+e^{-2b})^N}{2N!}\right)^{-1} - (\kappa_1 )^2                                                                                          \\
             & \rightarrow \frac{1}{2(e^{-4\beta}-1)^2 (N-1)!} \left(e^{6\beta} (2^N ) + e^{4\beta} (2^N )(N-2) + N(2^N ) + e^{2\beta} (-2^N )(2N-1) \right)                      \\
             & \left(\frac{2^N}{2N!}\right)^{-1} - (\kappa_1 )^2                                                                                                                  \\
             & = \frac{N(e^2\beta-1)^2 (e^2\beta+N)}{(e^{-4\beta}-1)^2} - (\kappa_1 )^2                                                                                           \\
             & = \frac{Ne^{8\beta} (e^2\beta+N)}{(e^{2\beta}+1)^2} - (\kappa_1 )^2                                                                                                \\
             & = N(1+N)/4-(N/2)^2=N/4, \, \kappa_2/\kappa_1 \rightarrow 0.5.
\end{align*}

\begin{align*}
    \text{For large } \tau & : d \rightarrow 0 \Rightarrow \beta \rightarrow \infty \Rightarrow \kappa_1 \rightarrow \frac{(2e^{-4\beta})/2(N-2)!}{1/(N)!} = N(N-1) e^{-4\beta},           \\
    \kappa_2               & \rightarrow \frac{(4e^{-4\beta})/2(N-2)!}{1/(N)!} - (N(N-1) e^{-4\beta})^2 = 2\kappa_1-\kappa_1^2, \, \kappa_2/\kappa_1 \rightarrow 2-\kappa_1 \rightarrow 2.
\end{align*}

\begin{align*}
    \text{For } N \rightarrow \infty               & :                                                                                                                   \\
    \sum_{n \, \text{even}} e^{-2\beta n}/n!(N-n)! & = \frac{1}{N!} \sum_{n=0}^{N/2} e^{-4\beta n} \binom{N}{2n}                                                         \\
                                                   & = \frac{1}{2N!} \left(\sum_{n=0}^N (-e^{-2\beta})^n \binom{N}{n} + \sum_{n=0}^N (e^{-2\beta})^n \binom{N}{n}\right) \\
                                                   & = \frac{(1-e^{-2b})^N+(1+e^{-2b})^N}{2N!} \rightarrow \frac{(1+e^{-2\beta})^N}{2N!}.
\end{align*}

\begin{align*}
    \sum_{n \, \text{even}} ne^{-2\beta n}/n!(N-n)! & = -\frac{(1 - e^{-2 b})^N  + (e^{-2 b}  + 1)^N  + e^{2 b}  ((1 - e^{-2 b})^N  - (e^{-2 b}  + 1)^N )}{2 (e^{4 b}  - 1)(N - 1)!} \\
                                                    & \rightarrow \frac{(e^{-2 b}  + 1)^N (e^{2 b}-1)}{2 (e^{4 b}  - 1)(N - 1)! }                                                    \\
                                                    & = \frac{(e^{-2 b}  + 1)^N}{2 (e^{2b}+ 1)(N - 1)! }.
\end{align*}

\begin{align*}
    \kappa_1 & = \frac{\sum_{n \, \text{even}} ne^{-2\beta n}/n!(N-n)!}{\sum_{n \, \text{even}} e^{-2\beta n}/n!(N-n)! } \\
             & \rightarrow \frac{(e^{-2 b}  + 1)^N 2N!}{2 (e^{2b}+ 1)(N - 1)!(1+e^{-2\beta})^N  }                        \\
             & = \frac{N}{1+e^{2\beta}} = Nd.
\end{align*}
\begin{align*}
     & \sum_{\substack{n                                                                                                                                                                      \\ \text{even}}} \frac{n^2 e^{-2\beta n}}{n!(N-n)!} = \frac{1}{2(e^{-4\beta}-1)^2 (N-1)!} \Biggl(e^{6\beta} ((1+e^{-2\beta})^N-(1-e^{-2\beta})^N) \\
     & \qquad + e^{4\beta} ((1+e^{-2\beta})^N+(1-e^{-2\beta})^N)(N-2) + N((1+e^{-2\beta})^N+(1-e^{-2\beta})^N)                                                                                \\
     & \qquad + e^{2\beta} ((1-e^{-2\beta})^N-(1+e^{-2\beta})^N)(2N-1) \Biggr)                                                                                                                \\
     & \rightarrow \frac{e^{6\beta} ((1+e^{-2\beta})^N) + e^{4\beta} ((1+e^{-2\beta})^N)(N-2) + N((1+e^{-2\beta})^N) + e^{2\beta} (-(1+e^{-2\beta})^N)(2N-1)}{2(e^{4\beta}-1)^2 (N-1)!}       \\
     & = \frac{(1+e^{-2\beta})^N (N+e^{2\beta})}{2(1+e^{2\beta})^2 (N-1)!}                                                                                                                    \\
     & \kappa_2 = \left(\frac{(1+e^{-2\beta})^N (N+e^{2\beta})}{2(1+e^{2\beta})^2 (N-1)!}\right) \bigg/ \left(\frac{(1+e^{-2\beta})^N}{2N!}\right) - \left(\frac{N}{1+e^{2\beta}}\right)^2    \\
     & = \frac{Ne^{2\beta}}{(1+e^{2\beta})^2} = Nd(1-d)                                                                                                                                       \\
     & \frac{\kappa_2}{\kappa_1} = \left(\frac{\sum_{\substack{n                                                                                                                              \\ \text{even}}} \frac{n^2 e^{-2\beta n}}{n!(N-n)!}}{\sum_{\substack{n \\ \text{even}}} \frac{ne^{-2\beta n}}{n!(N-n)!}}\right) - \left(\frac{\sum_{\substack{n \\ \text{even}}} \frac{ne^{-2\beta n}}{n!(N-n)!}}{\sum_{\substack{n \\ \text{even}}} \frac{e^{-2\beta n}}{n!(N-n)!}}\right) \\
     & \rightarrow \left(\frac{(1+e^{-2\beta})^N (N+e^{2\beta})}{2(1+e^{2\beta})^2 (N-1)!}\right) \bigg/ \left(\frac{(e^{-2 b} + 1)^N}{2 (e^{2b}+ 1)(N - 1)!}\right) - \frac{N}{1+e^{2\beta}} \\
     & = \frac{1}{1+e^{-2\beta}} = 1-d \approx 1- \frac{1}{2\pi \sqrt{(\hbar/2J\tau)}} \approx 1-0.1125\tau^{-1/2}.
\end{align*}

\section*{Notebook}
\begin{dmath}
2J\sum_{k>0} \left(g(t) \sin(k) (c_k^\dagger c_{-k}^\dagger + c_{-k} c_k) 
+ (1-g(t)-g(t)\cos(k))(c_k^\dagger c_k - c_{-k} c_{-k}^\dagger)\right)
\approx 
2J\sum_{k>0} \left(g(t) \sin(k) (( \langle c_k^\dagger \rangle \langle c_{-k}^\dagger \rangle + \langle c_k^\dagger \rangle c_{-k}^\dagger 
+ \langle c_{-k}^\dagger \rangle c_k^\dagger)+ c_{-k} c_k) 
+ (1-g(t)-g(t)\cos(k))(c_k^\dagger c_k - c_{-k} c_{-k}^\dagger)\right)
\end{dmath}
\end{document}
